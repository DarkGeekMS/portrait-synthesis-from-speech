على مر التاريخ ،كان فن البورتريه وسيلة للتعرف والعثور على الأفراد لأسباب مختلفة. بصرف النظر عن كونه فنًا لإحياء ذكرى الشخصيات التاريخية الشهيرة، فقد تم استخدامه لتحديد المجرمين والأفراد المفقودين. السبب الرئيسي وراء استخدام الصور الشخصية لهذه الأغراض هو أن البحث المرئي أسهل بكثير على البشر من أي نوع آخر من البحث. يمكن لمعظم الناس التعرف على شخص أو شيء ما، رأوه من قبل. حتى مع ظهور التصوير الفوتوغرافي، تشير الإحصائيات إلى أن الصور المرسومة لا تزال تستخدم في البحث المرئي. في الآونة الأخيرة، طورت البورتريهات تطبيقات أخرى أيضًا، لا سيما في دراسة الشخصيات والأحداث التاريخية.

على الرغم من أهميته، إلا أن رسم صورة كاملة للوجه يمكن أن يكون مرهقًا جدًا ويستغرق وقتًا طويلاً. تاريخياً، تطلبت هذه العملية رسام بورتريه محترف ويمكن أن تستغرق أسابيع لرسم صورة دقيقة بالكامل. حاليا ومع ظهور تطبيقات التصميم الرقمي مثل Adobe Photoshop، أصبحت هذه العملية أسهل بكثير. ومع ذلك، لا يزال الأمر يتطلب متخصصًا ويمكن أن يستغرق ساعات أو حتى أيامًا لإنجازه. علاوة على ذلك، قد تكون عملية تحسين الصورة المرسومة صعبة للغاية ويمكن أن تؤدي إلى إعادة الرسم في حالات معينة. من ناحية أخرى، يجب أن تكون عملية رسم البورتريه سريعة للغاية في بعض الأحيان، في حالات البحث عن مجرمين خطرين أو أطفال مفقودين.

يعالج ريتراتيستا مشكلات الوقت والدقة، مع الحفاظ على سهولة الاستخدام قدر الإمكان لغير المتخصصين. نحن نستخدم قوة النماذج التوليدية الحديثة لإنشاء نظام شامل لتوليد صورة الوجه من  وصف مجرد. يمكن أن يكون هذا الوصف صوتًا أو نصًا أو وصفًا يدويًا. نحن نعزل تعقيد نظامنا عن المستخدم من خلال توفير واجهة مستخدم ويب بسيطة. يستغرق نظامنا بضع ثوانٍ فقط لتكوين وجه كامل. كما أنه يسمح للمستخدم بتحسين الوجه الذي تم إنشاؤه بشكل أكبر وحتى عرضه في أوضاع متعددة لتوفير مزيد من التعريف. نقوم بتضمين مجموعة كافية نسبيًا من سمات الوجه لوصف الوجه البشري بشكل صحيح.
