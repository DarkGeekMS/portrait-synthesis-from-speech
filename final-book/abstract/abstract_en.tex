Throughout history, \emph{portraiture} has been a mean of identifying and finding individuals due to various reasons. Aside from being an art to memorialize famous historical characters, it has been used to identify criminals and missing individuals. The main reason behind using portraits for these purposes is that \emph{visual search} is much easier for humans than any other kind of search. Most people can identify a person or an object, they have seen before. Even with the rise of photography, statistics say that drawn portraits are still used in visual search. Recently, portraits developed other applications as well, especially in the study of historical characters and events.

While being important, sketching a complete face portrait can be very cumbersome and time-consuming. Historically, this process required a professional portraitist and could take him weeks to fully sketch an accurate portrait. However, with the rise of digital design application including \emph{Adobe Photoshop}, this process became much easier. Nonetheless, it still requires a professional and can take hours or even days to accomplish. Moreover, the process of refining a drawn portrait can be very tedious and can cause re-sketching in certain cases. On the other hand, the process of sketching a portrait sometimes needs to be extremely fast, in cases of searching for dangerous criminals or missing children. 

\textbf{Retratista} addresses both \emph{time} and \emph{accuracy} issues, while keeping the usage as simple as possible for non-professionals. We use the power of modern \emph{generative models} to create an \emph{end-to-end} system for \emph{face portrait generation} from bare description. This description can be a voice, text or manual description. We abstract the complexity of our system from the user by providing a simple \emph{web user interface}. Our system takes just a \emph{few seconds} to generate a complete face. Also, it allows the user to further \emph{refine} the generated face and even render it in \emph{multiple poses} to provide further identification. We include a relatively-sufficient set of \emph{facial attributes} to correctly describe a human face.
