Our project is completely software, so we discuss the software tools and packages used to develop our system.

\subsection{Software Tools}

\subsubsection{Ready-made Models}

\paragraph{Facial Landmark Detection Model}
The facial landmark detection model, used to extract some the feature directions, is completely ready-made. We used the open-source implementation of \textbf{High-resolution networks (HRNets) for facial landmark detection} \cite{sun2019highresolution} to do the landmark detection task, in order to get different facial distances and sizes.

\paragraph{Neural Face Renderer}
We used a ready-made neural network model for rendering $3D$ faces from $2D$ images to perform face rotation. This model is an open-source \emph{PyTorch} implementation for \textbf{Neural 3D Mesh Renderer} \cite{kato2017neural}.

\subsubsection{Frameworks and Packages}

\paragraph{PyTorch}
An highly-optimized open-source \emph{ML} library used to implement, train and deploy all \emph{DL} models used in our system.

\paragraph{OpenCV}
An open-source library for optimized image processing and computer vision applications. It's used in facial attributes classification in feature directions extraction.

\paragraph{Flask}
An open-source \emph{Python} framework, used to implement the server \emph{API} endpoints.

\paragraph{Vuejs}
An open-source \emph{JavaScript} framework, used to implement the web user interface.

\paragraph{Docker}
An open-source software used for containerization. It has been used throughout the project and through deployment for automation and to handle conflicting dependencies.
