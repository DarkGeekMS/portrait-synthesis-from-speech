Our system offers an innovative solution that leverages the power of \emph{Deep Learning}, \emph{Generative Adversarial Networks}, \emph{Computer Vision} and \emph{Natural Language Processing} to build an automated, fast and robust system for face generation and identification from various types of description. We push our pipeline to be as flexible and accurate as possible. Also we managed to overcome most of the challenges that faced us during building the system. Face generation and manipulation using voice, text and manual input are feasible in an automated way within few seconds. Moreover, our system considers a sufficient set of facial features to enable describing faces in an accurate way. Finally, our system is very flexible to add more facial features and further improve the results given enough resources and time.

\subsection{Faced Challenges}
Throughout the project, we faced many challenges. Basically, the challenges are related to \emph{data scarcity}, \emph{few related research work}, \emph{architectural complexity} and \emph{facial attributes}.

\subsubsection{Data Scarcity}
One of the main challenges of our project is data scarcity. No open-source source datasets exist for \emph{text-face} pairs, except only one dataset (\textbf{Face2Text}). We had to contact the authors, who kindly gave us an access for the dataset. Unfortunately, the dataset contains only $4000$ text-face pairs, which are not enough at all to conduct experiments or build the system. Consequently, we built our system in stages, as mentioned before. However, the \emph{text processing} module required a dataset, as it's completely supervised. Consequently, we built our face textual description from scratch, as discussed above \ref{sec:text}.

\subsubsection{Few Related Research Work}
Although the problem of image generation is popular in the research community, only one work actually targets human face generation from description. This work \cite{wang2020faces} does not even provide enough information about the method implementation. Consequently, we had to come up with our own working pipeline to build the face generation from description. We gained a lot of insights from other research works related to image (face) manipulation and awesome \texttt{StyleGAN2} follow-up papers.

\subsubsection{Architectural Complexity}
Most of the architectures related to high quality image generation are very complex and slow at inference. This results in the difficulty of editing or tuning them to other purposes. Also, we have to keep our inference time as minimal as possible. So, we had to spend a lot of time trying to adapt only the relevant modules of these enormous architectures into our system to maintain both time and memory complexity.

\subsubsection{Facial Attributes and Entanglement}
As mentioned before, the choice of facial attributes and the entanglement between them are the most difficult challenges of our project. We had to spend some time researching to select a set of facial attributes that correctly and sufficiently describes a human face. We came up with $32$ facial features for face generation and $38$ ($32$ + $6$ new) facial features for refinement. However, we were left with the entanglement challenge, because all open-source faces datasets contain real human faces. Real human faces contain natural entanglement between facial attributes. This results in features entanglement in the latent space of our generative model, which makes the process of feature directions extraction very difficult. We tried to solve this entanglement problem with various techniques and reached good results. However, the problem is not $100\%$ solved.

\subsection{Gained Experience}
Throughout our work on this project, we were exposed to numerous research works. We read lots of scientific research papers and articles, gained insights from many of them and adapted their techniques to our work. Also, we had the chance to work with multiple tools and technologies including, but not limited to, \emph{Python}, \emph{C/C++}, \emph{JavaScript}, \emph{PyTorch}, \emph{OpenCV}, \emph{REST APIs} and \emph{Docker}. Moreover, we gained a lot of experience from integrating such complex architectures into a single software.

\subsection{Conclusions}
Through the project, we managed to build a complete system that provides an automated, fast and robust way to generate and identify human faces from bare description. We consider voice, text and manual descriptions. Also, we formulated our system into a web application with a simple web user interface, where the user can input his preferred face description to be generated within few seconds. Then, the user is able to refine the generated face and render it in multiple poses for more identification. While being automated and accurate, our system suffers from some downsides as well. The facial features entanglement can still affect the output face portraits in some cases. Also, we tried to consider as much facial features as possible, however there can be even more features to include.

\subsection{Future Work}
Our work is flexible to changes and can be extended in various ways. To tackle the problem of entanglement, \texttt{StyleGAN2} synthesis network can be tuned using synthetic face images generated from \textbf{Epic's MetaHuman Creator}. This exposes the latent space to more generic faces with different facial features, which can help reducing entanglement and overlapping between feature clusters. Consequently, the set of facial features can be expanded to include more fine details about the human face. Following these methods, the system can be extended to generate fine-detailed human face portraits better than professional portraitists and within minimal time.
