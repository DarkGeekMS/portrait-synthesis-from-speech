As we described above, our users need the UI of the project to be as easy and familiar to them as possible, because they will use it in critical situations, such as a parent describes the criminal who kidnapped his/her child. So, we tried to make the both Front-end Design and Back-end Endpoints to be very simple and isolated from the complex techniques used in the background.

\subsubsection{Functional Description}

Our Application allows the user to get his/her intended face based on different generation modes that he/she can navigate through in the main page. Our generation modes are
\begin{itemize}
    \item Generation from Voice Description
    \item Generation from Textual Description
    \item Generation using Manual Manipulators
\end{itemize}
After each mode, the user is allowed to do extra sequential refinements using manual manipulators, and to generate 8 different head poses of the final refined face.


\subsubsection{Modular Decomposition}
As shown above, we can decompose this module into two sub-modules, Front-end Design and Back-end API

\paragraph{Front-end Design}
We used VueJS as our front-end framework with Axios as a request handler. Our front-end design consists of 6 pages

\begin{itemize}
    \item Main Page
    
    This Page is just a page with single list of generation modes as shown in figure \ref{fig:main}
    
    \begin{figure}[H]
        \centering
        \includegraphics[width=0.6\textwidth]{images/website/mainpage.jpg}
        \caption{Main Page}
        \label{fig:main}
    \end{figure}
    
    \item Generation From Voice Description Page
    
    This Page is a page with two different assets (voice and textual input) for the user to describe a face. After generation, it allows the user to refine the generated face or generate head poses, as shown in figure \ref{fig:voice}
    
    \begin{figure}[H]
        \centering
        \includegraphics[width=0.7\textwidth]{images/website/voice.png}
        \caption{Generation From Voice Description Page}
        \label{fig:voice}
    \end{figure}
    
    
    \item Generation From Textual Description Page
    
    This Page is a page with one asset (textual input) for the user to describe a face. After generation, it allows the user to refine the generated face or generate head poses, as shown in figure \ref{fig:text}
    
    \begin{figure}[H]
        \centering
        \includegraphics[width=0.7\textwidth]{images/website/text.png}
        \caption{Generation From Textual Description Page}
        \label{fig:text}
    \end{figure}
    
    \item Generation/Refinement Using Manual Manipulators Page
    
    This page allows the user to generate a new face or refine an actually-generated face using maual manipulators, as shown in figure \ref{fig:refine}
    
    \begin{figure}[H]
        \centering
        \includegraphics[width=0.7\textwidth]{images/website/refine1.png}
        \caption{Generation/Refinement Using Manual Manipulators Page}
        \label{fig:refine}
    \end{figure}
    
    \item Head Poses Page
    
    This page generates eight different head poses for extra identification, as shown in figure \ref{fig:poses}
    
    \item Help and Demo Page 
    
    This Page is a help aid for the user to know how to use the application and to support the user with some extra information about the project, such as
    \begin{itemize}
        \item Why is Retratista important?
        \item Statistics
        \item Project Features
        \item How To Use?
        \item Result Samples
        \item Team
    \end{itemize}
    
    \begin{figure}[H]
        \centering
        \includegraphics[width=0.7\textwidth]{images/website/poses.png}
        \caption{Head Poses Page}
        \label{fig:poses}
    \end{figure}
    
\end{itemize}


\paragraph{Back-end API}


\subsubsection{Design Constraints}

As described above, the main constraints on this module are

\begin{itemize}
    \item Simplicity of User Interface.
    \item Isolation of Complex Techniques used in background.
\end{itemize}